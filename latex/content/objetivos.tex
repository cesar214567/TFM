\section{Objetivos}

En este capítulo se presentan los objetivos y contribuciones 
de esta tesis. Una comprensión clara de estos aspectos es 
esencial para contextualizar el alcance y la relevancia de 
esta investigación. Las siguientes secciones ofrecerán una 
discusión detallada de los objetivos clave perseguidos en este 
trabajo, así como de las contribuciones que se pretende 
aportar al campo.

El objetivo principal de esta tesis es el diseño e 
implementación de un \textit{pipeline} que combina el uso de un 
extractor de características basado en CNN's junto con capas 
Bi-LSTM para la detección y clasificación de escenas violentas. 
Junto con ello, se planea optimizar la precisión y la velocidad 
del mismo haciendo múltiples experimentos modificando el 
clasificador y el número de celdas BI-LSTM
con el fin de construir el pipeline más efectivo para la 
detección automática de violencia en videos. Este objetivo 
surge de la necesidad de comprender cómo interactúan estos 
dos componentes fundamentales del aprendizaje profundo en 
tareas espaciotemporales complejas, como lo es la detección 
de violencia, y cómo su configuración afecta tanto la 
precisión como la eficiencia del sistema.

Adicionalmente, se plantean varios objetivos secundarios. 
En primer lugar, se busca realizar un preprocesado de datos 
que permita al \textit{pipeline} ser entrenado con ellos. 
En segundo lugar, se busca evaluar el impacto de diferentes 
arquitecturas CNN en la calidad de las características 
espaciotemporales extraídas, entendiendo cómo influyen 
estas variaciones en el rendimiento del modelo. Por último, 
se pretende investigar cómo la variación en el número 
de celdas LSTM afecta la modelación temporal y, por ende, 
la eficacia en la clasificación de eventos violentos.

\section{Contribuciones}

La contribución principal de esta tesis es el desarrollo 
de un pipeline optimizado para la detección de violencia 
que equilibra la complejidad de la extracción de 
características basada en CNN y el número de celdas LSTM, 
con el objetivo de lograr una mayor precisión y eficiencia. 
Al analizar sistemáticamente los compromisos entre estos 
dos componentes, este trabajo proporciona un enfoque 
estructurado para el diseño de arquitecturas de aprendizaje 
profundo orientadas al reconocimiento espaciotemporal de 
patrones violentos en video.

Este enfoque contribuye significativamente al avance del 
análisis de video asistido por inteligencia artificial 
para aplicaciones de vigilancia en tiempo real. A través 
del diseño y evaluación de múltiples configuraciones 
arquitectónicas, se demuestra cómo el ajuste preciso de 
los módulos CNN y LSTM puede resultar en modelos más 
efectivos y menos costosos computacionalmente. Esta tesis, 
por tanto, ofrece una guía práctica para investigadores y 
desarrolladores interesados en implementar soluciones 
robustas y escalables en escenarios del mundo real donde 
la detección temprana de violencia es crítica.
