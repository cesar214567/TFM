
\section{Goals}
In this chapter, the objectives and contributions of this 
thesis will be presented. A clear understanding of these 
aspects is essential to contextualize the scope and significance 
of this research. The following sections will provide a 
detailed discussion of the key goals pursued in this work, as 
well as the contributions it aims to make to the field.
\begin{itemize}
  \item { Main goal: 
      \begin{itemize}
          \item optimize the trade-off between modifying 
          the CNN feature extractor and adjusting the number 
          of LSTM cells to construct the most effective pipeline 
          for violence detection
      \end{itemize}
   }
   \item { Objetivos secundarios:
      \begin{itemize}
          \item Assess the impact of various CNN architectures 
          on the quality of extracted spatiotemporal features 
          for violence detection.
          \item Investigate how varying the number of LSTM cells 
          affects temporal modeling and classification performance.
          \item Identify the optimal balance between CNN feature 
          extraction complexity and LSTM capacity to achieve the 
          best performance with minimal computational cost.
          \item Automate the labeling process by creating a 
          real-time application for the pipeline.
      \end{itemize}
      }
\end{itemize}

\section{Contributions}

The main contribution of this thesis is the development of 
an optimized pipeline for violence detection that balances 
the complexity of CNN-based feature extraction and the number 
of LSTM cells to achieve superior accuracy and efficiency. 
By systematically analyzing the trade-offs between these two 
components, this work provides a structured approach to 
designing deep learning architectures for spatiotemporal 
violence recognition, improving both detection performance 
and computational feasibility. This contribution aims to 
advance AI-driven video analysis for real-time surveillance 
applications.