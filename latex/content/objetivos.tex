\section{Objetivos}
En este capítulo se presentarán los objetivos y contribuciones de 
esta tesis. Una comprensión clara de estos aspectos es 
esencial para contextualizar el alcance y la relevancia de 
esta investigación. Las siguientes secciones ofrecerán una 
discusión detallada de los objetivos clave perseguidos en 
este trabajo, así como de las contribuciones que se pretende 
aportar al campo.
\begin{itemize}
  \item { Objetivo principal: 
      \begin{itemize}
          \item optimizar el equilibrio entre la modificación 
          del extractor de características CNN y el ajuste del 
          número de celdas LSTM para construir el pipeline más 
          efectivo para la detección de violencia
      \end{itemize}
   }
   \item { Objetivos secundarios:
      \begin{itemize}
          \item Evaluar el impacto de diversas arquitecturas 
          CNN en la calidad de las características 
          espaciotemporales extraídas para la detección de 
          violencia.
          \item Investigar cómo la variación en el número 
          de celdas LSTM afecta la modelación temporal y el 
          rendimiento en la clasificación.
          \item Identificar el equilibrio óptimo entre la 
          complejidad de la extracción de características por 
          CNN y la capacidad de las LSTM para lograr el mejor 
          rendimiento con un costo computacional mínimo.
          \item Automatizar el proceso de etiquetado mediante la 
          creación de una aplicación en tiempo real para el pipeline.
      \end{itemize}
      }
\end{itemize}

\section{Contribuciones}

La contribución principal de esta tesis es el desarrollo de 
un pipeline optimizado para la detección de violencia que 
equilibra la complejidad de la extracción de características 
basada en CNN y el número de celdas LSTM para lograr una 
mayor precisión y eficiencia. Al analizar sistemáticamente 
los compromisos entre estos dos componentes, este trabajo 
proporciona un enfoque estructurado para diseñar arquitecturas 
de deep learning destinadas al reconocimiento espaciotemporal 
de violencia, mejorando tanto el rendimiento en la detección 
como la viabilidad computacional. Esta contribución tiene 
como objetivo avanzar en el análisis de video impulsado por 
IA para aplicaciones de vigilancia en tiempo real.
