Este trabajo tiene como propósito evaluar el desempeño de 
distintas arquitecturas híbridas CNN-Bi-LSTM para la detección 
automática de violencia en vídeos, utilizando el 
\textit{dataset Hockey Fights}. Se diseñó un \textit{pipeline} 
de clasificación que combina extractores de características 
basados en redes neuronales convolucionales (ResNet50, 
EfficientNetB0, EfficientNetV2 y MobileNetV3) con capas 
bidireccionales de memoria a corto y largo plazo (Bi-LSTM), 
variando su cantidad de 1 a 9 capas. Cada combinación fue 
entrenada y evaluada utilizando métricas estándar como precisión, 
pérdida y tiempo de inferencia. Los resultados mostraron que 
todas las configuraciones alcanzaron un \textit{accuracy} de 
prueba del 99\%, aunque con diferentes configuraciones de 
celdas Bi-LSTM. Al comparar todas las métricas, EfficientNetB0 
fue la arquitectura más estable, logrando el mejor equilibrio 
entre rendimiento, eficiencia computacional y capacidad de 
generalización. Estos hallazgos son relevantes para aplicaciones 
en análisis de vídeo en tiempo real y sistemas de vigilancia 
inteligentes.

{\bf Palabras Clave:} detección de violencia, Redes Neuronales Convolucionales,
Bidirectional-Long Short Term Memory, EfficientNetB0, 
clasificación de sequencias, análisis de video.
