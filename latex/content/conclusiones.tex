A lo largo del desarrollo experimental de este trabajo, se 
exploraron diversas configuraciones arquitectónicas con el 
objetivo de determinar la combinación óptima entre complejidad 
estructural y rendimiento en la tarea de detección de violencia 
en videos del \textit{dataset Hockey Fights}. Las pruebas 
realizadas incluyeron variantes en los modelos de CNN utilizados 
para la extracción de características y el número de unidades LSTM. 
No obstante, a pesar de estas diferencias en la arquitectura, 
los resultados obtenidos de las mejores configuraciones para cada 
CNN en el conjunto de prueba fueron notablemente consistentes.\\

Todos los modelos evaluados alcanzaron un desempeño muy similar en 
términos del \textit{accuracy}, con un valor promedio de 98.75\%, lo que 
pone de manifiesto la solidez del enfoque adoptado. Este 
comportamiento sugiere que la calidad de las representaciones 
generadas por la red convolucional utilizada como extractor de 
características (EfficientNetB0) resulta suficientemente 
informativa para que incluso una configuración secuencial mínima 
(como 5 celdas Bi-LSTM) sea capaz de capturar de manera 
efectiva la dinámica temporal relevante para la clasificación. 
En consecuencia, se concluye que el sistema propuesto no solo 
ofrece un alto rendimiento en términos de precisión, sino que 
también presenta ventajas en cuanto a eficiencia computacional y 
simplicidad, factores especialmente valiosos para futuras 
implementaciones en entornos con recursos limitados o 
aplicaciones en tiempo real.\\

Como línea de trabajo futuro, se propone profundizar en el 
análisis de arquitecturas más complejas que integren mecanismos 
de atención o variantes modernas de redes recurrentes, tales 
como GRU o Transformers, con el fin de evaluar si estos enfoques 
pueden aportar mejoras adicionales en tareas de clasificación 
de video más exigentes o en contextos con mayor variabilidad 
visual. Asimismo, sería pertinente validar la robustez del 
modelo actual frente a escenarios del mundo real mediante su 
evaluación en \textit{datasets} más extensos, heterogéneos y no balanceados, 
que representen condiciones variadas de iluminación, movimiento 
de cámara, resolución y presencia de ruido. También se considera 
de interés explorar técnicas de aprendizaje auto-supervisado 
o semi-supervisado que permitan aprovechar grandes volúmenes de 
datos no etiquetados, con el objetivo de reducir la dependencia 
de anotaciones manuales. Finalmente, la implementación del 
sistema en dispositivos con capacidad limitada de cómputo (como 
cámaras inteligentes o sistemas embebidos) podría abrir nuevas 
oportunidades de aplicación práctica, siempre que se garantice 
un equilibrio adecuado entre rendimiento, eficiencia y capacidad 
de inferencia en tiempo real.