


Violence remains a critical global issue, with millions of 
incidents reported annually. According to the World Health 
Organization (WHO), interpersonal violence has remained as one 
of the top 10 causes of deaths per year in the Americas regions, 
sin with countless more cases of physical aggression and violent 
crimes going unreported \cite{WorldHealthOrganization2024}. 
Latin America, in particular, has some of the highest violence rates, 
with countries like Peru, Chile, Brazil, Colombia, and Mexico experiencing 
significant challenges in crime prevention and public security 
\cite{Bisca2024}. In 2024, the National Institute of Statistics and 
Geography (INEGI) reported 21.9 million legal-age victims only on 
Mexico, and 31.3 million crimes \cite{INEGI2024}. The rising 
availability of video surveillance and digital media presents 
an opportunity to develop automated systems capable of detecting 
and mitigating violent incidents in real time.

Artificial intelligence (AI) has emerged as a powerful tool 
in the field of video analysis, offering promising solutions 
for automated violence detection. Deep learning models, 
particularly convolutional neural networks (CNNs) and 
long-short term models (LSTM's), have demonstrated 
exceptional capabilities in processing spatiotemporal 
features from video data \cite{Orozco2021}. By leveraging 
these technologies, AI-based systems can analyze video 
streams, recognize violent actions, and trigger alerts 
with high accuracy. However, challenges such as class 
imbalance, data scarcity, and false positives remain 
critical hurdles in real-world applications\cite{Kulkarni2021}. This 
research aims to enhance the robustness and interpretability 
of AI-driven violence detection systems, contributing to 
safer environments in Mexico and beyond.

\section{Justification}

The escalating rates of violence in Latin America, 
particularly in Mexico as exposed in the prior section, 
underscore the urgent need for advanced 
surveillance systems capable of real-time incident detection. 
Traditional monitoring approaches, which depend on human 
oversight, are often inefficient due to cognitive fatigue and 
limitations in scalability\cite{Marois2021}. Artificial intelligence (AI), 
particularly deep learning, has demonstrated significant 
potential in automating violence detection through the 
integration of convolutional neural networks (CNNs) and 
long short-term memory (LSTM) networks
\cite{Negre2024,Negre20242,Abdali2019,Sharma2021}. While CNNs extract 
spatial features from video frames, LSTMs capture temporal 
dependencies, making them a powerful combination for analyzing 
dynamic scenes. However, the optimal design of these models 
remains an open challenge, as variations in CNN architectures 
and LSTM configurations directly affect detection accuracy, 
computational efficiency, and real-world applicability.

This study seeks to systematically investigate the trade-off 
between different CNN feature extractors and the number of LSTM 
cells to determine the most effective pipeline for violence 
detection. The choice of CNN influences feature extraction 
quality, while the number of LSTM cells impacts the model's 
ability to capture temporal patterns without incurring 
excessive computational costs. By optimizing this balance, 
the research aims to improve both the performance and 
efficiency of AI-driven violence detection systems. The 
outcomes will contribute not only to the academic advancement 
of spatiotemporal video analysis but also to the practical 
deployment of robust and scalable surveillance solutions, 
ultimately enhancing public safety in Mexico and beyond.

\section{Work Approach}

Having established the justification for this research, 
it is evident that the selection of feature extraction 
techniques and the number of LSTM cells play a crucial 
role in optimizing violence detection models. Existing 
approaches often overlook the trade-off between these 
two factors, potentially limiting performance in real-world 
applications. Therefore, this study aims to evaluate and 
refine the balance between CNN feature extractors and LSTM 
cell configurations to develop a more efficient and accurate 
pipeline for automated violence detection.

\section{Document Structure}

to be defined