La violencia sigue siendo un problema global crítico, con millones 
de incidentes reportados anualmente. Según la Organización Mundial 
de la Salud (OMS), la violencia interpersonal ha permanecido como 
una de las 10 principales causas de muerte anual en las regiones 
de las Américas, con innumerables casos de agresión física y 
delitos violentos que no se reportan \cite{WorldHealthOrganization2024}. 
América Latina, en particular, tiene algunas de las tasas de violencia 
más altas, con países como Perú, Chile, Brasil, Colombia y México 
enfrentando importantes desafíos en la prevención del crimen y 
la seguridad pública \cite{Bisca2024}. En 2024, el Instituto Nacional 
de Estadística y Geografía (INEGI) reportó 21.9 millones de víctimas 
mayores de edad solo en México, y 31.3 millones de delitos \cite{INEGI2024}. 
La creciente disponibilidad de videovigilancia y medios digitales 
presenta una oportunidad para desarrollar sistemas automatizados 
capaces de detectar y mitigar incidentes violentos en tiempo real.

La inteligencia artificial (IA) ha emergido como una herramienta 
potente en el campo del análisis de video, ofreciendo soluciones 
prometedoras para la detección automatizada de violencia. Los modelos 
de aprendizaje profundo, particularmente las redes neuronales 
convolucionales (CNNs) y las redes de memoria a largo y corto plazo 
(LSTM), han demostrado capacidades excepcionales en el procesamiento 
de características espaciotemporales de los datos de video \cite{Orozco2021}. 
Aprovechando estas tecnologías, los sistemas basados en IA pueden 
analizar flujos de video, reconocer acciones violentas y generar alertas 
con alta precisión. Sin embargo, desafíos como el desequilibrio de clases, 
la escasez de datos y los falsos positivos siguen siendo obstáculos 
críticos en las aplicaciones del mundo real \cite{Kulkarni2021}. Esta 
investigación tiene como objetivo mejorar la robustez e interpretabilidad 
de los sistemas de detección de violencia impulsados por IA, contribuyendo 
a entornos más seguros en México y más allá.

\section{Justificación}

Las tasas crecientes de violencia en América Latina, 
particularmente en México como se expuso en la sección anterior, 
subrayan la urgente necesidad de sistemas avanzados de 
videovigilancia capaces de detectar incidentes en tiempo real. 
Los enfoques tradicionales de monitoreo, que dependen de la supervisión 
humana, a menudo son ineficientes debido a la fatiga cognitiva y 
las limitaciones en la escalabilidad \cite{Marois2021}. La inteligencia 
artificial (IA), particularmente el aprendizaje profundo, ha demostrado 
un gran potencial para automatizar la detección de violencia mediante 
la integración de redes neuronales convolucionales (CNNs) y redes de 
memoria a largo y corto plazo (LSTM) \cite{Negre2024,Negre20242,Abdali2019,Sharma2021}. 
Mientras que las CNNs extraen características espaciales de los 
fotogramas de video, las LSTMs capturan dependencias temporales, 
lo que las convierte en una combinación poderosa para analizar 
escenas dinámicas. Sin embargo, el diseño óptimo de estos modelos 
sigue siendo un desafío abierto, ya que las variaciones en las 
arquitecturas de CNN y las configuraciones de LSTM afectan directamente 
la precisión de la detección, la eficiencia computacional y la aplicabilidad 
en el mundo real.

Este estudio busca investigar sistemáticamente el balance 
entre diferentes extractores de características de CNN y el número 
de celdas LSTM para determinar la pipeline más efectiva para la 
detección de violencia. La elección de la CNN influye en la calidad 
de la extracción de características, mientras que el número de celdas 
LSTM impacta en la capacidad del modelo para capturar patrones temporales 
sin incurrir en costos computacionales excesivos. Al optimizar este 
balance, la investigación busca mejorar tanto el rendimiento como 
la eficiencia de los sistemas de detección de violencia impulsados por IA. 
Los resultados contribuirán no solo al avance académico del análisis 
espaciotemporal de video, sino también al despliegue práctico de soluciones 
de videovigilancia robustas y escalables, mejorando finalmente la seguridad 
pública en México y más allá.

A continuación se especifican tanto el enfoque de este 
trabajo y cual será la estructura del resto del 
documento.

\section{Enfoque de Trabajo}

Habiendo establecido la justificación para esta investigación, 
es evidente que la selección de técnicas de extracción de 
características y el número de celdas LSTM juegan un papel crucial 
en la optimización de los modelos de detección de violencia. Los enfoques 
existentes a menudo pasan por alto el balance entre estos dos factores, 
lo que puede limitar el rendimiento en aplicaciones del mundo real. 
Por lo tanto, este estudio tiene como objetivo evaluar y refinar 
el balance entre los extractores de características de CNN y las 
configuraciones de celdas LSTM para desarrollar una pipeline más 
eficiente y precisa para la detección automatizada de violencia.

\section{Estructura del Documento}

Este documento se organiza en varios apartados que permiten 
comprender el desarrollo completo del trabajo. Primero, se 
expone el contexto del problema y su relevancia. Luego, se 
definen los objetivos de la investigación. La sección de 
metodología describe los datos utilizados, el preprocesamiento 
y el enfoque técnico adoptado. A continuación, en el desarrollo 
del trabajo, se detallan los experimentos realizados, las 
configuraciones evaluadas y los resultados obtenidos. Finalmente, 
se presentan las conclusiones y se proponen posibles líneas de 
trabajo futuro.